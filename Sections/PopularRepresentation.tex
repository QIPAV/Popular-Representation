\section{Popular Representation}\\\\\\
Quadcopters are currently a hot topic in today’s society and in technological development. They have a wide range of applications and can perform a variety of tasks, such as aerial photography, acrobatic flying, transportation of goods and many more. Quadcopters are currently used by military, law enforcements, hobbyists and for commercial applications. 
\\\\
Over the last few years we have seen a significant reduction in cost related to quadcopters. This has made the technology more widely available and contributed to a leap in 
advances and new applications. But beside all the research done, there is marginal documentation on the benefits of quadcopters using variable pitch systems.
\\\\
From helicopters we know that variable pitch is essential for maneuverability. The purpose of the variable pitch system is to shift the thrust produced by the propeller blades quickly, generating the required lift. The vast majority of helicopters are solely dependent on variable pitch to maneuver because they operate on a constant RPM. \cite{name}
\\\\
In comparison, most multicopters have fixed pitch propellers. This means that their movement is solely controlled by accelerating or decelerating individual propellers to change position. The drawback to this is that all movement involves acceleration, which in turn draws a lot more power than it would operating on constant RPM and only changing the pitch angle. 
\\\\
But what if a multicopter had the ability to control both pitch and RPM? Would this yield superior agility and maneuverability, and what effects will this have on efficiency? 
\\\\
On this basis forsvarets forskningsinstitutt (FFI) has requested that we investigate the advantages and disadvantages of variable pitch quadcopters. 

